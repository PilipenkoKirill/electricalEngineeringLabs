\documentclass[14pt,a4paper]{article}
% \usepackage{cmap} %Улучшает поиск по pdf документу
%%%%%%%%%%%%%%Пользовательские команды%%%%%%%%%
\usepackage{latexsym,amsmath,amssymb,amsbsy,graphicx}
\usepackage{icomma}
\usepackage{tikz}
\usepackage{pgfplots}
\usepackage[american]{circuitikz}
\usepackage{mathtext} 				% русские буквы в формулах
\usepackage{caption}
\usepackage{subfigure}
\usepackage[version=4]{mhchem} % the canonical chemistry package (example: \ce{^{32}_{15}P})
\usepackage{graphicx}
\graphicspath{{images/}}
\DeclareGraphicsExtensions{.pdf,.png,.jpg}
%%%%%%%%%%%%%%%%%%%%%%%%Оформление по ГОСТУ
\usepackage{fontspec}
\setmainfont[Renderer=Basic,Ligatures={TeX}]{Times New Roman}
\usepackage[english,russian]{babel} %Поддержка русской локализации
\usepackage[14pt]{extsizes} % для того чтобы задать нестандартный 14-ый размер шрифта
\usepackage{indentfirst} %Задаёт отступ самого первого абзаца
\setlength\parindent{1.25cm}
\usepackage[a4paper, left=3cm, top=1.5cm, right=1.5cm, bottom=2cm]{geometry}
\usepackage{setspace}
%\sloppy %Выравнивание текст по ширине и решение проблемы переполнением строки
\onehalfspacing %Полуторный интервал
\usepackage{mathtext} % русские буквы в формулах
%%%%%%%%%%%%%%%%%%%%%%%%%%%%%
\begin{document}
\begin{center}
  \textbf{\large Лабораторная работа №5} \vspace{1cm}

  \textbf{\Large Исследование переходных процессов в RC-цепях}
\end{center}
  
\textbf{\underline{Цель работы:}} исследовать RC-цепи при подаче на их вход однополярных прямоугольных импульсов различной частоты; получить на экране осциллографа осциллограммы выходных сигналов и определить по полученным осциллограммам частоту среза RC-цепи.

\textbf{\underline{Приборы и принадлежности:}}

\begin{enumerate}
\def\labelenumi{\arabic{enumi}.}
\item Осциллограф GOS-620,
\item Мультиметр АРРА-62,
\item Генератор прямоугольных импульсов,
\item Монтажная плата для сборки RC-цепей (одна и та же плата используется при выполнении лабораторных работ №4 и №5),
\item Соединительные провода.
\end{enumerate}

\textbf{Контрольные вопросы}

\begin{enumerate}
\def\labelenumi{\arabic{enumi}.}
\item Запишите закон изменения напряжения на конденсаторе и резисторе при
  зарядке конденсатора от источника постоянного напряжения через
  резистор R, если конденсатор предварительно был разряжен. Представьте
  соответствующие графики.
\item Запишите закон изменения напряжения на конденсаторе и резисторе при разрядке конденсатора через резистор R, если конденсатор был полностью
  заряжен. Представьте соответствующие графики.
\item
  Нарисуйте схемы интегрирующей и дифференцирующей RC - цепей.
\item
  Объясните формулы (1) и (9) из раздела «Краткая теория».
\item
  Запишите и объясните условия интегрирования и дифференцирования
  напряжения с помощью RC-цепей.
\item
  Объясните, почему RC-цепь с резистором на выходе называется
  дифференцирующей, а RC-цепь с конденсатором на выходе --  интегрирующей.
\item
  Что такое скважность импульсов, относительная длительность импульсов.
\item
  Какой электрический сигнал называют меандр.
\item
  Как фильтрующие свойства RC--цепей влияют на форму выходных сигналов,
  если на вход подаётся сигнал типа меандр.
\item
  Что такое частота среза для RC-цепи и как она связана с постоянной
  времени RC-цепи.
\item
  Зарисуйте в тетрадях экспериментальные установки для наблюдения
  переходных процессов в RC-цепях.
\end{enumerate}

\textbf{Краткая теория}

Простейшая RC-цепь представляет собой последовательно соединённые
резистор и конденсатор (рис.\ref{DifAndIntCirc}). Цепь с резистором на выходе называется дифференцирующей RC-цепью, с конденсатором -- интегрирующей RC-цепью.

Исследуем переходные процессы в RC-цепях при подаче на вход
прямоугольного однополярного импульса, такого, что время импульса \(t_\text{и}\) равно времени паузы (рис.\ref{variousSignals} и рис. \ref{ResistInten}). Такой сигнал называют
меандр.

\begin{figure}[htbp]
  \begin{minipage}{.48\textwidth}
  \centering
  \begin{circuitikz}
    \draw (0,3) to[C=C,o-] (3,3) to[short,-o] (4,3);
    \draw (0,0) to[short,o-o] (4,0);
    \draw(3,0)  to[generic, *-*, l=R] (3,3);
    \draw[<->] (0,0.05) -- (0,2.95) node [pos=.5, anchor=east] {$U_\text{вх.}$};
    \draw[<->] (4,0.05) -- (4,2.95) node [pos=.5, anchor=west] {$U_\text{вых.}$};
  \end{circuitikz} \\
  Дифференцирующая RC-цепь
\end{minipage}\hfill
\begin{minipage}{.48\textwidth}
  \begin{circuitikz}
    \draw (0,3) to[generic=R,o-] (3,3) to[short,-o] (4,3);
    \draw (0,0) to[short,o-o] (4,0);
    \draw(3,0)  to[C=C, *-*,] (3,3);
    \draw[<->] (0,0.05) -- (0,2.95) node [pos=.5, anchor=east] {$U_\text{вх.}$};
    \draw[<->] (4,0.05) -- (4,2.95) node [pos=.5, anchor=west] {$U_\text{вых.}$};
  \end{circuitikz} \\
  Интегрирующая RC-цепь
\end{minipage}
  \caption{}
  \label{DifAndIntCirc}
\end{figure}

\emph{\textbf{Переходные процессы в интегрирующей RC-цепи}}

При подаче прямоугольного импульса на вход интегрирующей RC-цепи конденсатор заряжается при наличии импульса и разряжается во время
паузы, формируя таким образом сигнал на выходе (см.рис. \ref{variousSignals}). На формы выходного напряжения влияет величина τ -- постоянная времени RC-цепи.
Рассмотрим данный вопрос подробнее.

\begin{figure}[htbp]
  \centering
  \begin{tikzpicture}
    \draw[->] (0,0) -- (0,2) node [anchor=east] {U\textsubscript{вх.}};
    \draw[->] (0,0) -- (4.5,0) node [anchor=north] {$t$};
    \draw[thick, red] (0,1) foreach \x in {0,2} {-- (\x,1) -- (\x+1,1) -- (\x+1,0) -- (\x+2,0) };
  \end{tikzpicture}
  \begin{tikzpicture}
    \draw[->] (0,0) -- (0,2) node [anchor=east] {$U_\text{вых.}$};
    \draw[->] (0,0) -- (4.5,0) node [anchor=north] {$t$};
    \foreach \i in {0,2} {
    \draw[domain=0:1, variable=\x, smooth, red, thick, shift=({\i,0})] plot ({\x}, {1-e^(-\x/.12)});
    \draw[domain=0:1, variable=\x, smooth, red, thick, shift=({\i+1,0})] plot ({\x}, {e^(-\x/.12)});}
    \node[at={(4.5,2)},anchor=north east] {$\tau\ll t_\text{и}$};
  \end{tikzpicture}
  \begin{tikzpicture}
    \draw[->] (0,0) -- (0,2) node [anchor=east] {$U_\text{вых.}$};
    \draw[->] (0,0) -- (4.5,0) node [anchor=north] {$t$};
    \foreach \i in {0,2} {
    \draw[domain=0:1, variable=\x, smooth, red, thick, shift=({\i,0})] plot ({\x}, {1-e^(-\x/.3)});
    \draw[domain=0:1, variable=\x, smooth, red, thick, shift=({\i+1,0})] plot ({\x}, {(1-e^(-1/.3))*e^(-\x/.3)});}
    \node[at={(4.5,2)},anchor=north east] {$\tau = t_\text{и}$};
  \end{tikzpicture}
  \begin{tikzpicture}
    \draw[->] (0,0) -- (0,2) node [anchor=east] {$U_\text{вых.}$};
    \draw[->] (0,0) -- (4.5,0) node [anchor=north] {$t$};
    \foreach \i in {0,2} {
    \draw[domain=0:1, variable=\x, smooth, red, thick, shift=({\i,0})] plot ({\x}, {1-e^(-\x/1.2)});
    \draw[domain=0:1, variable=\x, smooth, red, thick, shift=({\i+1,0})] plot ({\x}, {(1-e^(-1/1.2))*e^(-\x/.2)});}
    \node[at={(4.5,2)},anchor=north east] {$\tau \gg t_\text{и}$};
  \end{tikzpicture}
  \caption{}
  \label{variousSignals}
\end{figure}

Запишем выражения для вычисления тока в данной цепи:
\begin{equation} \label{eq1}
  i = \frac{U_{R}}{R} = \frac{U_\text{вх.} - U_\text{вых.}}{R} = C\frac{dU_\text{вых.}}{dt}.
\end{equation}
Из (\ref{eq1}) получим:
\begin{equation} \label{eq2}
  dU_\text{вых.} = \frac{1}{RC}\left( U_\text{вх.} - U_\text{вых.} \right)dt. 
\end{equation}
Приведём уравнение (\ref{eq2}) к виду:
\begin{equation}
  \frac{dU_\text{вых.}}{dt} + \frac{1}{RC}\left( U_\text{вых.} - U_\text{вх.} \right) = 0.
\end{equation}
Выходное напряжение является напряжением на конденсаторе: \(U_\text{вых.} = U_{C}.\) Тогда, получим
\begin{equation} \label{eq4}
  \frac{{dU}_{C}}{dt} + \frac{1}{RC}\left( U_{C} - U_\text{вх.} \right) = 0.
\end{equation}
Перепишем равенство (\ref{eq4}):
\begin{equation*}
\frac{{dU}_{C}}{dt} = \frac{1}{RC}\left( U_\text{вх.} - U_{C} \right)
\end{equation*}
и проинтегрируем его левую и правую части:
\begin{equation} \label{eq5}
  U_{C} = \frac{1}{RC}\int \left( U_\text{вх.} - U_{C} \right)dt = \frac{1}{RC}\int U_\text{вх.}dt - \frac{1}{RC}\int U_{C}dt.
\end{equation}

Рассмотрим 2 случая использования формулы (\ref{eq5}):
\begin{enumerate}
  \item \(U_{C} \ll \frac{1}{RC}\int U_{C}dt\), значит  \(\frac{1}{RC}\int U_\text{вх.}dt - \frac{1}{RC}\int U_{C}dt \approx 0\),
  и, соответственно \(U_{C} \approx U_\text{вх.}.\) Очевидно, сигнал на выходе
  приближённо повторяет входной сигнал. Для выполнения этого условия
  необходимо, чтобы τ имела малую величину \(\tau \ll t_\text{и}\) (рис. \ref{variousSignals}).
  \item \(U_{C} \gg \frac{1}{RC}\int U_{C}dt\), значит \(U_{C} \approx \frac{1}{RC}\int U_\text{вх.}dt\ \)~---~видно, что
  выходное напряжение зависит от интеграла входного напряжения (отсюда и
  название -- интегрирующая RC-цепь). В данном случае цепь хорошо выполняет свою интегрирующую функцию, и чем   будет больше величина τ, тем интегрирующие свойства будут лучше, т.е
  \(\tau \gg t_\text{и}\) (рис. \ref{variousSignals}).
\end{enumerate}

Для объяснения формы выходных осциллограмм необходимо рассмотреть зарядно-разрядные процессы в цепи с конденсатором.

\emph{\textbf{Зарядно-разрядные процессы в цепи с конденсатором.}}

При подаче импульса на вход цепи конденсатор будет
\emph{\textbf{заряжаться.}} При этом напряжение на конденсаторе не может
измениться скачком. Получим формулу напряжения на конденсаторе при его
зарядке, решив дифференциальное уравнение (\ref{eq4}) и положив, что в момент
подачи импульса на вход цепи напряжение на конденсаторе будет равно
нулю: \(U_{C}(0) = 0\). Решением уравнения (\ref{eq4}) будет функция:
\begin{equation} \label{eq6}
  U_{C}(t) = U_\text{вх.}\left( 1 - e^{- \frac{t}{\tau}} \right),
\end{equation}
где $\tau$~---~постоянная времени RC-цепи
\[\tau = R \cdot C, \]
а формула (\ref{eq6}) является формулой напряжения на конденсаторе в момент времени $t$ при его зарядке, при условии, что в начальный момент времени конденсатор был полностью разряжен. График напряжения конденсатора при его зарядке представлен на рисунке \ref{CondenInten}-a. Имея такой график, можно
определить постоянную времени RC-цепи $\tau$, а так же частоту среза
\(\nu_\text{ср}\), учитывая, что при $t=\tau$ формула (\ref{eq6}) принимает вид:
\(U_{C}(t) = U_\text{вх.}\left( 1 - e^{- 1} \right) = 0,63U_\text{вх.}\).

\begin{figure}[htbp]
  \centering
  \begin{tikzpicture}
    \draw[->] (0,0) -- (0,3.5) node [anchor=west] {U};
    \draw[->] (0,0) node [anchor=north east] {0} -- (4.5,0) node [anchor=north] {$t$};
    \draw[dashed] (0,0.63*3) -- (2.2,0.63*3) node [pos=0,anchor=east] {63\%};
    \draw[dashed] (2.2,0) -- (2.2,0.63*3) node [pos=0,anchor=north] {$\tau$};
    \draw[dashed, thick] (0,2.7) -- (4.5,2.7) node [pos=0,anchor=east] {U\textsubscript{вх.}};
    \draw[domain=0:4.5, variable=\x, smooth, red, thick] plot ({\x}, {3*(1-e^(-\x/2.2))});
    \node[at={(4.5,1.7)},anchor=north east] {а)};
  \end{tikzpicture}
  \begin{tikzpicture}
    \draw[->] (0,0) -- (0,3.5) node [anchor=west] {U};
    \draw[->] (0,0) node [anchor=north east] {0} -- (4.5,0) node [anchor=north] {$t$};
    \draw[dashed] (0,0.37*3) -- (2.2,0.37*3) node [pos=0,anchor=east] {37\%};
    \draw[dashed] (2.2,0) -- (2.2,0.37*3) node [pos=0,anchor=north] {$\tau$};
    \draw[domain=0:4.5, variable=\x, smooth, red, thick] plot ({\x}, {3*(e^(-\x/2.2))});
    \node[at={(4.5,1.7)},anchor=north east] {б)};
  \end{tikzpicture}
  \caption{Изменение напряжения конденсатора при его зарядке~(а) и разрядке~(б)}
  \label{CondenInten}
\end{figure}

Если на входе цепи импульса нет, конденсатор будет
\emph{\textbf{разряжаться.}} Но напряжение на конденсаторе не может уменьшиться скачком. Получим формулу напряжения на конденсаторе при его
разрядке, решив дифференциальное уравнение (\ref{eq4}) и учитывая, что в момент
отключения импульса на входе цепи \(U_{C}(0) = U_\text{вх.}.\) Решением уравнения (\ref{eq4}) будет в этом случае функция:
\begin{equation} \label{eq7}
  U_{C}(t) = U_\text{вх.}e^{- \frac{t}{\tau}}.
\end{equation}

Формула (\ref{eq7}) является формулой напряжения на конденсаторе при его разрядке, если в начальный момент конденсатор был заряжен до напряжения
\(U_\text{вх.}\). Если \emph{t=τ} формула (7) примет вид:
\(U_{C}(t) = U_\text{вх.}e^{- 1} = 0,37U_\text{вх.}\). График напряжения конденсатора при его разрядке представлен на рисунке \ref{CondenInten}-б. Имея такой график, можно определить постоянную времени RC-цепи $\tau$, а так же частоту среза $\nu_\text{ср}$.

\emph{\textbf{Переходные процессы в дифференцирующей RC-цепи}}

При подаче прямоугольного импульса на вход дифференцирующей цепи
конденсатор заряжается и в соответствии с (6) напряжение на нём растёт.
Напряжение на резисторе при этом уменьшается. Во время паузы конденсатор
начинает разряжаться через источник напряжения в соответствии с формулой
(7). При этом напряжение на резисторе растёт в обратной полярности
(рис. \ref{ResistInten}). Рассмотрим данный вопрос подробнее.

\begin{figure}[htbp]
  \centering
  \begin{tikzpicture}[>=stealth]
    \draw[->] (0,0) -- (0,2) node [anchor=east] {U\textsubscript{вх.}};
    \draw[->] (0,0) -- (4.5,0) node [anchor=north] {$t$};
    \draw[thick, red] (0,1) foreach \x in {0,2} {-- (\x,1) -- (\x+1,1) -- (\x+1,0) -- (\x+2,0) };
  \end{tikzpicture}
  \begin{tikzpicture}[>=stealth]
    \draw[->] (0,-1) -- (0,2) node [anchor=east] {$U_\text{вых.}$};
    \draw[->] (0,0) -- (4.5,0) node [anchor=north] {$t$};
    \foreach \i in {0,2} {
    \draw[domain=0:1, variable=\x, smooth, red, thick, shift=({\i,0})] plot ({\x}, {e^(-\x/2)});
    \draw[domain=0:1, variable=\x, smooth, red, thick, shift=({\i+1,0}), rotate around x=180,
    ] plot ({\x}, {e^(-\x/2)});}
    \node[at={(4.5,2)},anchor=north east] {$\tau\gg t_\text{и}$};
  \end{tikzpicture}
  \begin{tikzpicture}[>=stealth]
    \draw[->] (0,-1) -- (0,2) node [anchor=east] {$U_\text{вых.}$};
    \draw[->] (0,0) -- (4.5,0) node [anchor=north] {$t$};
    \foreach \i in {0,2} {
    \draw[domain=0:1, variable=\x, smooth, red, thick, shift=({\i,0})] plot ({\x}, {e^(-\x/.35)});
    \draw[domain=0:1, variable=\x, smooth, red, thick, shift=({\i+1,0}), rotate around x=180,
    ] plot ({\x}, {e^(-\x/.35)});}
    \node[at={(4.5,2)},anchor=north east] {$\tau = t_\text{и}$};
  \end{tikzpicture}
  \begin{tikzpicture}[>=stealth]
    \draw[->] (0,-1) -- (0,2) node [anchor=east] {$U_\text{вых.}$};
    \draw[->] (0,0) -- (4.5,0) node [anchor=north] {$t$};
    \foreach \i in {0,2} {
    \draw[domain=0:1, variable=\x, smooth, red, thick, shift=({\i,0})] plot ({\x}, {e^(-\x/.06)});
    \draw[domain=0:1, variable=\x, smooth, red, thick, shift=({\i+1,0}), rotate around x=180,
    ] plot ({\x}, {e^(-\x/.06)});}
    \node[at={(4.5,2)},anchor=north east] {$\tau \ll t_\text{и}$};
  \end{tikzpicture}
  \caption{Напряжение на резисторе в RC-цепи}
  \label{ResistInten}
\end{figure}

Выходное напряжение дифференцирующей RC-цепи равно напряжению на
резисторе:
\begin{equation} \label{eq8}
  U_\text{вых.} = U_\text{вх.} - U_{C} = i \cdot R.
\end{equation}
Ток конденсатора прямо пропорционален скорости изменения напряжения, приложенного к нему:
\begin{equation} \label{eq9}
  i = C\frac{{dU}_{C}}{dt}.
\end{equation}
С учётом (\ref{eq8}) из (\ref{eq9}) получим:
\begin{equation} \label{eq10}
  U_\text{вых.} = CR\frac{{dU}_{C}}{dt} = CR\left( \frac{{dU}_\text{вх}}{dt} - \frac{dU_\text{вых.}}{dt} \right) = CR\frac{{dU}_\text{вх}}{dt} - CR\frac{dU_\text{вых.}}{dt}.
\end{equation}

Рассмотрим 2 случая использования формулы (\ref{eq10}):
\begin{itemize}
\item
  \(CR\frac{dU_\text{вых.}}{dt} \ll U_\text{вых.}\), значит
  \(U_\text{вых.} \approx \ CR\frac{{dU}_\text{вх}}{dt}\). Очевидно, что напряжение
  на выходе представляет из себя дифференциал входного сигнала (отсюда и
  название -- дифференцирующая RC-цепь). Можно заметить, что данное
  условие лучше выполняется при малых значениях \(\tau\), и, чем меньше
  будет $\tau$, тем лучше будет дифференцирование (рис. \ref{ResistInten}).
\item \(CR\frac{dU_\text{вых.}}{dt} \gg U_\text{вых.}\), значит
  \(CR\frac{{dU}_\text{вх}}{dt} = CR\frac{dU_\text{вых.}}{dt}\) или \(U_\text{вых.} = U_\text{вх.}\). Данное условие выполняется при больших значениях
  τ (рис. \ref{ResistInten}).
\end{itemize}

\textbf{Лабораторные задания.}

\begin{enumerate}
\item Выберите на монтажной плате резистор и конденсатор. Пользуясь мультиметром АРРА-62 определите сопротивление резистора и ёмкость
  конденсатора. \emph{\textbf{Внимание! При выполнении   лабораторных работ №4 и №5 используйте только одну пару элементов
  резистор-конденсатор.}} Определите постоянную времени для выбранной
  RC-цепи.
\item Соберите цепь для исследования переходных процессов в дифференцирующей RC-цепи при подаче на его вход прямоугольных импульсов напряжения (рис. \ref{DifToOscilloscope}).

\begin{figure}[hptb]
  \begin{minipage}{0.48\textwidth}
    \centering
    \begin{circuitikz}
      \draw (0,3) to[C=C,o-] (3,3) to[short,-o] (4,3);
      \draw (0,0) to[short,o-o] (4,0);
      \draw(3,0) node[ground]{} to[generic, *-*, l=R] (3,3) ;
      \draw[<->] (4,0.05) -- (4,2.95) node [pos=.5, anchor=west] {$U_\text{вых.}$};
      \draw[{Circle}->] (1,2.94) -- (1,4) node[anchor=east] {Y1};
      \draw[{Circle}->] (3.5,2.94) -- (3.5,4) node[anchor=east] {Y2};
      %Поступаемый сигнал
      \draw[->] (-2,1) -- (-2,2) node [anchor=east] {$U$};
      \draw[->] (-2,1) -- (0.5,1) node [anchor=north] {$t$};
      \draw[thick, red] (-2,1.6) foreach \x in {-2,-1} {-- (\x,1.6) -- (\x+.5,1.6) -- (\x+.5,1) -- (\x+1,1) };
    \end{circuitikz}
    \caption{}\label{DifToOscilloscope}
  \end{minipage}\hfill
  \begin{minipage}{0.48\textwidth}
    \centering
    \begin{circuitikz}
      \draw (0,3) to[generic=R,o-] (3,3) to[short,-o] (4,3);
      \draw (0,0) to[short,o-o] (4,0);
      \draw(3,0) node[ground]{} to[C=C, *-*,] (3,3);
      \draw[<->] (4,0.05) -- (4,2.95) node [pos=.5, anchor=west] {$U_\text{вых.}$};
      \draw[{Circle}->] (.7,2.94) -- (.7,4) node[anchor=east] {Y1};
      \draw[{Circle}->] (3.5,2.94) -- (3.5,4) node[anchor=east] {Y2};
      %Поступаемый сигнал
      \draw[->] (-2,1) -- (-2,2) node [anchor=east] {$U$};
      \draw[->] (-2,1) -- (0.5,1) node [anchor=north] {$t$};
      \draw[thick, red] (-2,1.6) foreach \x in {-2,-1} {-- (\x,1.6) -- (\x+.5,1.6) -- (\x+.5,1) -- (\x+1,1) };
    \end{circuitikz}
    \caption{}\label{IntToOscilloscope}
  \end{minipage}
\end{figure}

\item Перерисуйте в тетрадь осциллограммы напряжений на выходе
  дифференцирующей RC-цепи не менее чем для трёх различных длительностей
  прямоугольных импульсов напряжения (постоянная времени RC-цепи
  значительно больше длительности импульса, постоянная времени RC-цепи
  примерно равна длительности импульса, постоянная времени RC-цепи
  значительно меньше длительности импульса).
\item
  Соберите цепь для исследования переходных процессов в интегрирующей
  RC-цепи при подаче на его вход прямоугольных импульсов напряжения
  (рис. \ref{IntToOscilloscope}).
\item
  Перерисуйте в тетрадь осциллограммы напряжений на выходе интегрирующей
  RC-цепи не менее чем для трёх различных длительностей прямоугольных
  импульсов напряжения (постоянная времени RC-цепи значительно больше
  длительности импульса, постоянная времени RC-цепи примерно равна
  длительности импульса, постоянная времени RC-цепи значительно меньше
  длительности импульса).
\item
  По полученным в заданиях 3 и 5 осциллограммам определите постоянную
  времени RC-цепи, и сравните полученное значение с расчетным.
\end{enumerate}

\textbf{\underline{Литература.}}

\begin{enumerate}
\def\labelenumi{\arabic{enumi}.}
\item Основы   промышленной электроники: Учеб. пособие для неэлектротехн. спец. вузов/ В.Г. Герасимов, О.М. Князьков, А.Е. Краснопольский, В.В. Сухоруков; Под ред. В.Г. Герасимова. - М.: Высш. шк., 1986. - 336 с.
\item Ляшко М.Н. Радиотехника: Лаб. практикум. - Мн.: Выш. школа, 1981. - 269 с.
\item Манаев Е.И. Основы радиоэлектроники: Учеб. пособие для вузов. - М.: Радио и связь, 1985. - 488 с.
\item Инструкции к измерительным приборам.
\end{enumerate}
\end{document}
